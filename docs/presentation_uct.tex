\documentclass{beamer}
\usepackage{multirow}
\usepackage{graphicx}
\usepackage{amsmath}
\usepackage{lscape}
\usepackage{rotating}
\usepackage{setspace}
\usepackage{sidecap}
\usepackage{times}
\usepackage[utf8x]{inputenc}
\usepackage[T1]{fontenc}
\usepackage[french]{babel}
\usepackage{amsmath}
\usepackage{amsfonts}
\usepackage{multirow}
\usepackage{bigstrut}
\usepackage{color}
\usepackage[normalem]{ulem}
\usepackage{ifthen}
\usepackage{cancel}
\input{rgb.tex}

\newcommand{\gclass}[1]{\textcolor{compl}{\textrm{#1}}}


\useinnertheme{rounded}
\usecolortheme{orchid}
\useinnertheme{circles}

\setbeamercolor{titlelike}{fg=darkred,bg=white}
\setbeamercolor{itemize item}{fg=darkred}
\setbeamercolor{subitem}{fg=darkred}
\setbeamercolor{item projected}{bg=darkred}

\mode<presentation>

\setbeamercolor{frametitle}{bg=gray!15!white}
\setbeamercolor*{author in head/foot}{parent=palette tertiary}
\setbeamercolor*{title in head/foot}{parent=palette secondary}
\setbeamercolor*{date in head/foot}{parent=palette quaternary,fg=black}

\setbeamercolor*{section in head/foot}{parent=frametitle,fg=black}
\setbeamercolor*{subsection in head/foot}{parent=palette primary}

\setbeamertemplate{navigation symbols}
{
}

\setbeamertemplate{footline}
{
  \leavevmode%
  \hbox{%
%   \begin{beamercolorbox}[wd=.333333\paperwidth,ht=2.25ex,dp=1ex,center]{author in head/foot}%
%     \usebeamerfont{author in head/foot}\insertshortauthor~~(\insertshortinstitute)
%   \end{beamercolorbox}%
%   \begin{beamercolorbox}[wd=.333333\paperwidth,ht=2.25ex,dp=1ex,center]{title in head/foot}%
%     \usebeamerfont{title in head/foot}\insertshorttitle
%   \end{beamercolorbox}%
  \begin{beamercolorbox}[wd=\paperwidth,ht=2.25ex,dp=1ex,right]{date in head/foot}%
    \usebeamerfont{date in head/foot}
    \bf
    \insertframenumber{} / \inserttotalframenumber\hspace*{2ex}
  \end{beamercolorbox}}%
  \vskip0pt%
}

\setbeamertemplate{headline}
{
  \leavevmode%
  \hbox{%
  \begin{beamercolorbox}[wd=\paperwidth,ht=2.25ex,dp=1ex,left]{section in head/foot}%
    \usebeamerfont{section in head/foot}
    \bf
    \hspace*{2ex}\insertsectionhead\hspace*{2ex}\insertsubsectionhead
  \end{beamercolorbox}}%
  \vskip0pt%
}

\setbeamersize{text margin left=1em,text margin right=1em}

% \AtBeginSection[]{
%  \frame<handout:0>{
%    \frametitle{Plan}
%    \tableofcontents[current]
%  }
% }

\mode<all>


\title{Artificial Intelligence for deterministic 2 players games with UCT}
\author{Pierre Gueth}
\institute{}
\date{\today}

\DeclareMathOperator{\corr}{corr}

\begin{document}

\maketitle

% \begin{frame}
% \frametitle{Simulation}
% \begin{itemize}
%  \item
% \end{itemize}
% \end{frame}

\section{Introduction}

\begin{frame}
\frametitle{Motivation}
\begin{block}{}
\begin{itemize}
\item Artificial intelligence 
\item PG camera
\item Discrepancies between planned and actual configurations
\end{itemize}
\end{block}
\begin{block}{Aims}
\begin{itemize}
\item \alert{Prediction of perturbation amplitude}
\item Study of correlation between PG profiles and perturbation components
\begin{itemize}
\item Distal falloff position
\item Maximum correlation between profiles
\end{itemize}
\end{itemize}
\end{block}
\end{frame}

\begin{frame}
\frametitle{Context}
% \begin{block}{Protontherapy}
% \begin{itemize}
% \item Physics validated experimentally for dose deposition
% \item Proton experimental PG data
% \item Internal Monte Carlo models need tuning for heavy projectiles
% \end{itemize}
% \end{block}
\begin{block}{Granularity}
\begin{itemize}
\item Whole treatment fraction ($\sim10^{10}$ protons for $2~\mathrm{Gy}$)
\item Energy layer (constant energy, $\sim10^{9}$ protons for $2~\mathrm{Gy}$)
\item \alert<2->{Spot by spot monitoring} ($\sim10^7$ protons for $2~\mathrm{Gy}$)
\end{itemize}
\end{block}
\begin{block}{Monitoring device}
\begin{itemize}
%\item PET ring (offline, limited by annihilation)
\item Vertex imaging (inline, only for heavy projectiles)
\item \alert<3->{Prompt gamma camera} (inline, proton and heavy projectiles)
\end{itemize}
\end{block}
\end{frame}

\begin{frame}
\frametitle{PG monitoring technology}
\begin{block}{Compton camera}
\begin{itemize}
 \item Compton interaction in scatter detector
 \item Emission position from line-cone intersection
\end{itemize}
\end{block}
\begin{block}{Knife edge}
\begin{itemize}
 \item Count hits of PG passing through a knife edge slit
 \item Easy geometrical reconstruction
 %\item Low counting rate
\end{itemize}
\end{block}
\begin{block}{\alert<2>{Multiple slits}}
\begin{itemize}
 \item Count hits of PG passing through slits
 \item Straightforward reconstruction
 %\item Higher counting rate than pinhole
\end{itemize}
\end{block}
\end{frame}

\section{Method}

\begin{frame}
\frametitle{Types of perturbation}
\begin{itemize}
 \item \alert<2>{Patient position}
 \item Patient orientation
 \item Stoichiometric calibration of TPS
 \item Tumor regression / anatomical change
 \item Internal organ motion
\end{itemize}
\end{frame}

\begin{frame}
\frametitle{Classifier training}
\begin{itemize}
 \item Find measure threshold by maximizing associated specificity and sensitivity (MASS)
\end{itemize}
\vspace*{.5cm}
$$\operatorname{specificity} = \frac{\mathrm{TN}}{\mathrm{TN}+\mathrm{FP}}$$
$$\operatorname{sensitivity} = \frac{\mathrm{TP}}{\mathrm{TP}+\mathrm{FN}}$$
$$\operatorname{MASS} = \max_{MT} \sqrt{\operatorname{specificity}^2 + \operatorname{sensitivity}^2}$$
\begin{itemize}
 \item Upper-left most point in ROC curves
\end{itemize}
\end{frame}

\begin{frame}
\frametitle{Classifier training}
\begin{center}
\end{center}
\end{frame}

\section{Results}

\begin{frame}
\begin{center}
\frametitle{C++ implementation}
\includegraphics[width=\linewidth]{board_virtual}
\end{center}
\end{frame}

\begin{frame}
\begin{center}
\frametitle{Google AI Challenge 2010: Tron}
\includegraphics[width=\linewidth]{tron}
finished 50th over more than 3000 contestans.
\end{center}
\end{frame}

\section{Conclusion}

\begin{frame}
\begin{center}
\Huge Demonstation \\ Can you beat the IA?
\end{center}
\end{frame}

\begin{frame}
\begin{center}
\Huge Thanks for your attention
\end{center}
\end{frame}

\begin{frame}
\frametitle{Pierre Gueth}
\begin{block}{Wide technical and academical knowledge}
\begin{itemize}
\item Classe préparatoire PT
\item ENS Cachan \\ aggrégation physique appliquée / EEA
\item Thèse au laboratoire CREATIS (Université Lyon I) \\ Imagerie médicale ultrasonore \\ Estimation de mouvement
\item Post doc au Centre Léon Bérard (Lyon) \\ Simulation Monte-Carlo Protonthérapie (GATE) \\ Imagerie $\gamma$-prompt
\end{itemize}
\end{block}
\begin{block}{Numerous computer side project}
\begin{itemize}
\item Freesiege, Blocks, ...
\item UCT
\item Autojump, cluster submission tools
\end{itemize}
\end{block}
\end{frame}

\end{document}
